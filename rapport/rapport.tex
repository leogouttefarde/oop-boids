\documentclass [a4paper,11pt,titlepage] {article}
\usepackage{graphicx}
\usepackage{pdfpages}
\usepackage{fancybox}
\usepackage[francais]{babel}
\usepackage[utf8]{inputenc}
% \usepackage[T1]{fontenc}
\usepackage{amsmath,amsfonts,amssymb}
\usepackage{fancyhdr}
\usepackage{stackrel}
\usepackage{xspace}
\usepackage{url}
\usepackage{titling}
\usepackage{listings}
\usepackage{color}

\definecolor{dkgreen}{rgb}{0,0.6,0}
\definecolor{gray}{rgb}{0.5,0.5,0.5}
\definecolor{mauve}{rgb}{0.58,0,0.82}

\lstset{frame=tb,
  language=Java,
  aboveskip=3mm,
  belowskip=3mm,
  showstringspaces=false,
  columns=flexible,
  basicstyle={\small\ttfamily},
  numbers=none,
  numberstyle=\tiny\color{gray},
  keywordstyle=\color{blue},
  commentstyle=\color{dkgreen},
  stringstyle=\color{mauve},
  breaklines=true,
  breakatwhitespace=true,
  tabsize=3
}

\setlength{\parindent}{0pt}
\setlength{\parskip}{1ex}
\setlength{\textwidth}{17cm}
\setlength{\textheight}{24cm}
\setlength{\oddsidemargin}{-.7cm}
\setlength{\evensidemargin}{-.7cm}
\setlength{\topmargin}{-.5in}


\predate{
\begin{center}
}
\postdate{
\\
\vspace{1.5cm}
\includegraphics[scale=0.7]{imag.png}
\end{center}}


\title {{ {\huge Compte rendu du projet}} \\
``{\em Simulation Orientée-Objet de systèmes multiagents}'' }

\author {Equipe 32 \\
\\
DOUMA Nejmeddine\\
GOUTTEFARDE Léo\\
KACHER Ilyes}
\date{Lundi 16 Novembre 2015}
% \date{Lundi 16 Novembre 2015\endgraf\bigskip
% Equipe NN}

\lhead{Projet POO}
\rhead{Compte rendu}

\begin{document}
\pagestyle{fancy}
\maketitle

\begin{center}
\section* {Introduction }
\end{center}


L’objectif de ce TP était d’implémenter en \texttt{Java} des simulations
graphiques de quelques systèmes multiagents. Un des points les plus importants de ce TP était d'aborder les aspects fondamentaux de la programmation orientée objet. Nous avons implémenter trois simulateurs de systèmes de type automate cellulaire: le jeu de la vie de Conway, un jeu de l'immigration et le modèle de ségrégation de Schelling en plus d'un simulateur d’un système de mouvement d’essaims auto-organisés : le modèle de Boids.


$\newline$
\section {Choix conceptuels}
\subsection{Organisation en package}
Afin de faciliter la navigation à travers les différentes classes du projet, nous les avons classé dans des différents package:
\begin{itemize}
\item \texttt{element} : contient les classes qui implémentent les différents agents (\texttt{Boid}, \texttt{Lighter}, \texttt{Predator} et \texttt{Prey})
\item \texttt{event} : contient l'\texttt{EventManager} et les différentes classes pour gérer les différents types d'\texttt{event}
\item \texttt{group} : contient les différentes classes qui implémentent les groupes des agents
\item \texttt{simulator} : contient toutes les classes simulator qui réalisent \texttt{Simulable}
\item \texttt{test} : contient tous les test
\item \texttt{utility} : contient des classes divers utilisés par les autres classes du projet
\end{itemize}
\subsection{Un premier simulateur: Balls}
Pour réaliser ce simulateur nous avons suivi le sujet qui a été assez guidé. Cette étape nous a permis de nous familiarisé avec la librairie \texttt{gui.jar} fournie et en particulier l'interface \texttt{Simulable}.
\subsection{Automates cellulaires}
Nous avons cherché à factoriser notre code tout en profitant des avantages de la programmation orientée objet en implémentant deux classes abstraites \texttt{Automaton} et \texttt{AutomatonSimulator}. Pour chaque système d'automate cellulaire simulé il y a une classe concrète qui hérite la classe \texttt{Automaton} et implémente les règles du système et une classe qui le simule en héritant \texttt{AutomatonSimulator}. Nous avons remarqué plusieurs similarités entre l'implémentation du jeux de l'immigration et du modèle du Schelling, donc, dans le but de plus factoriser notre code, nous avons définie une autre classe abstraite \texttt{ExtendedAutomaton} qui étend \texttt{Automaton} et qui est héritée par les implémentations de ces deux systèmes.

\begin{center}
\begin{tabular}{|l|c|c|c|}
  \hline
    Système & Implémentation & Simulateur \\
  \hline
   Jeux de la vie de Conway & \texttt{Life} & \texttt{LifeSimulator}\\
  \hline
  Jeux de l'immigration & \texttt{Immigration} & \texttt{ImmigrationSimulator}\\
  \hline
  Modèle de Schelling & \texttt{Schelling} & \texttt{SchellingSimulator}\\
  \hline
 
\end{tabular}
\end{center}

Nous avons aussi généralisé l'utilisation de \texttt{EventManager} pour le calcul des générations suivantes pour tous les systèmes d'automate cellulaire simulés en définissant \texttt{AutomatonEvent} qui hérite la classe abstraite \texttt{Event}.


\subsection{Mouvement d’essaims auto-organisés: les Boids}
\subsubsection {Le système de boids}
Nous avons utilisé presque la même architecture que pour le simulateur des balles. La classe \texttt{Boid} implémente les règles de l'agent d'un boid, la classe \texttt{Boids} implémente un environnement de plusieurs boids tandis que le simulateur du système se trouve dans la classe \texttt{BoidsSimulator}.
\subsubsection {Le système Proies/Prédateurs}

Les classes \texttt{Predator} et \texttt{Prey} étendent la classe \texttt{Boid} et implémentent les règles nécessaires pour chaque population.

\section {Tests}

\subsection {Exécution}
Tous les classes test sont dans le package \texttt{test} et sont repatis selont le tableau suivant:

\begin{center}
\begin{tabular}{|l|c|c|}
  \hline
    Classe & Système simulé & Commande pour l'exécution \\
  \hline
  \texttt{TestGUI} & Test de l'interface graphique & \texttt{make exeGUI}\\
  \hline
   \texttt{TestBalls} & Test demandé à la première question & \\
  \hline
  \texttt{TestBallsSimulator} & Balls & \texttt{make exeBSim}\\
  \hline
  \texttt{TestLifeSimulator} & Jeux de la vie de Conway & \texttt{make exeLifeSim}\\
  \hline
  \texttt{TestImmigrationSimulator} & Jeu de l'immigration & \texttt{make exeImmSim}\\
  \hline
  \texttt{TestSchellingSimulator} & Modèle de Shelling & \texttt{make exeSSim}\\
  \hline
  \texttt{TestEventManager} & Test de l'event manager & \texttt{make exeEvents}\\
  \hline
  \texttt{TestBoidsSimulator} & Boids simples & \texttt{make exeBoids}\\
  \hline
  \texttt{TestPreyPredatorSimulator} & Boids proies/prédateurs  & \texttt{make exePred}\\
  \hline
  
 
\end{tabular}
\end{center}

\subsection {Résultats}
\subsubsection {Jeux de la vie de Conway}
BLA BLA BLA

\subsubsection {Jeux de l'immigration}
BLA BLA BLA

\subsubsection {Modèle de Schelling}
BLA BLA BLA

\subsubsection {Boids}
BLA BLA BLA

\subsubsection {Système Proie/Prédateur}
BLA BLA BLA

\begin{center}
\section* {Conclusion }
\end{center}

Ce projet nous a permis de nous confronter aux problématiques du modèle orienté objet et comment profiter de ces aspects fondamenteaux pour produire un code à la fois factorisé et réutilisable. Il nous a permis non seulement de mieux comprendre les notions du cours
mais aussi de découvrir le monde des simulateurs multi-agents. Le
projet était intéressant et il a permis d’avoir un rendu visuel, ce qui est fort agréable.
\end{document}




% 4 pages max
% Explique / justifie choix conception / bonne utilisation des classes et méthodes
% Peut également décrire principaux tests effectués et résultats obtenus

