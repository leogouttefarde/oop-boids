\documentclass [a4paper,11pt,titlepage] {article}
\usepackage{graphicx}
\usepackage{pdfpages}
\usepackage{fancybox}
\usepackage[francais]{babel}
\usepackage[utf8]{inputenc}
% \usepackage[T1]{fontenc}
\usepackage{amsmath,amsfonts,amssymb}
\usepackage{fancyhdr}
\usepackage{stackrel}
\usepackage{xspace}
\usepackage{url}
\usepackage{titling}
\usepackage{listings}
\usepackage{color}

\definecolor{dkgreen}{rgb}{0,0.6,0}
\definecolor{gray}{rgb}{0.5,0.5,0.5}
\definecolor{mauve}{rgb}{0.58,0,0.82}

\lstset{frame=tb,
  language=Java,
  aboveskip=3mm,
  belowskip=3mm,
  showstringspaces=false,
  columns=flexible,
  basicstyle={\small\ttfamily},
  numbers=none,
  numberstyle=\tiny\color{gray},
  keywordstyle=\color{blue},
  commentstyle=\color{dkgreen},
  stringstyle=\color{mauve},
  breaklines=true,
  breakatwhitespace=true,
  tabsize=3
}

\setlength{\parindent}{0pt}
\setlength{\parskip}{1ex}
\setlength{\textwidth}{17cm}
\setlength{\textheight}{24cm}
\setlength{\oddsidemargin}{-.7cm}
\setlength{\evensidemargin}{-.7cm}
\setlength{\topmargin}{-.5in}


\predate{
\begin{center}
}
\postdate{
\\
\vspace{1.5cm}
\includegraphics[scale=0.7]{imag.png}
\end{center}}


\title {{ {\huge Compte rendu du projet}} \\
``{\em Simulation Orientée-Objet de systèmes multiagents}'' }

\author {Equipe 32 \\
\\
DOUMA Nejmeddine\\
GOUTTEFARDE Léo\\
KACHER Ilyes}
\date{Lundi 16 Novembre 2015}
% \date{Lundi 16 Novembre 2015\endgraf\bigskip
% Equipe NN}

\lhead{Projet POO}
\rhead{Compte rendu}

\begin{document}
\pagestyle{fancy}
\maketitle

\begin{center}
\section* {Introduction }
\end{center}


L’objectif de ce TP était d’implémenter en \texttt{Java} des simulations
graphiques de quelques systèmes multiagents. Un des points les plus importants de ce TP était d'aborder les aspects fondamentaux de la programmation orientée objet. Nous avons implémenter trois simulateurs de systèmes de type automate cellulaire: le jeu de la vie de Conway, un jeu de l'immigration et le modèle de ségrégation de Schelling en plus d'un simulateur d’un système
de mouvement d’essaims auto-organisés : le modèle de Boids.


$\newline$
\section {Choix conceptuels}
\subsection{Automates cellulaires}
Nous avons cherché à factoriser notre code tout en profitant des avantages de la programmation orientée objet en implémentant deux classes abstraites \texttt{Automaton} et \texttt{AutomatonSimulator}. Pour chaque système d'automate cellulaire simulé il ya une classe concrète qui hérite la classe \texttt{Automaton} et implémente les règles du système et une classe qui le simule en héritant \texttt{AutomatonSimulator}. Nous avons remarqué plusieurs similarités entre l'implémentation du jeux de l'immigration et du modèle du Schelling, donc, dans le but de plus factoriser notre code, nous avons définie une autre classe abstraite \texttt{ExtendedAutomaton} qui étend \texttt{Automaton} et qui est hérité par les implémentations de ces deux systèmes.

\begin{tabular}{|l|c|c|c|}
  \hline
    Système & Implémentation & Simulateur & Test \\
  \hline
   Jeux de la vie de Conway & \texttt{Life} & \texttt{LifeSimulator} & \texttt{TestLifeSimulator}\\
  \hline
  Jeux de l'immigration & \texttt{Immigration} & \texttt{ImmigrationSimulator} & \texttt{TestLifeSimulator}\\
  \hline
  Modèle de Schelling & \texttt{Schelling} & \texttt{SchellingSimulator} & \texttt{TestSchellingSimulator}\\
  \hline
 
\end{tabular}

Nous avons aussi généralisé l'utilisation de \texttt{EventManager} pour le calcul des générations suivantes pour tous les systèmes d'automate cellulaire simulés en définissant \texttt{AutomatonEvent} qui hérite la classe abstraite \texttt{Event}.

\subsubsection {Jeux de la vie de Conway}
Pour visualiser l'exécution de ce simulateur, il faut exécuter la commande: \texttt{make exeLifeSim}

\subsubsection {Jeux de l'immigration}
Pour visualiser l'exécution de ce simulateur, il faut exécuter la commande: \texttt{make exeImmSim}

\subsubsection {Modèle de Schelling}
Pour visualiser l'exécution de ce simulateur, il faut exécuter la commande: \texttt{make exeSSim}
\subsection{Mouvement d’essaims auto-organisés: les Boids}

\section {Tests}

blabla tests


\end{document}




% 4 pages max
% Explique / justifie choix conception / bonne utilisation des classes et méthodes
% Peut également décrire principaux tests effectués et résultats obtenus

