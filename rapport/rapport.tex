\documentclass [a4paper,11pt,titlepage] {article}
\usepackage{graphicx}
\usepackage{pdfpages}
\usepackage{fancybox}
\usepackage[francais]{babel}
\usepackage[utf8]{inputenc}
% \usepackage[T1]{fontenc}
\usepackage{amsmath,amsfonts,amssymb}
\usepackage{fancyhdr}
\usepackage{stackrel}
\usepackage{xspace}
\usepackage{url}
\usepackage{titling}
\usepackage{listings}
\usepackage{color}
\usepackage{array}
\usepackage{makecell}
\newcolumntype{x}[1]{>{\centering\arraybackslash}p{#1}}
\usepackage{tikz}
\newcommand\diag[4]{%
  \multicolumn{1}{p{#2}|}{\hskip-\tabcolsep
  $\vcenter{\begin{tikzpicture}[baseline=0,anchor=south west,inner sep=#1]
  \path[use as bounding box] (0,0) rectangle (#2+2\tabcolsep,\baselineskip);
  \node[minimum width={#2+2\tabcolsep},minimum height=\baselineskip+\extrarowheight] (box) {};
  \draw (box.north west) -- (box.south east);
  \node[anchor=south west] at (box.south west) {#3};
  \node[anchor=north east] at (box.north east) {#4};
 \end{tikzpicture}}$\hskip-\tabcolsep}}

\definecolor{dkgreen}{rgb}{0,0.6,0}
\definecolor{gray}{rgb}{0.5,0.5,0.5}
\definecolor{mauve}{rgb}{0.58,0,0.82}

\lstset{frame=tb,
  language=Java,
  aboveskip=3mm,
  belowskip=3mm,
  showstringspaces=false,
  columns=flexible,
  basicstyle={\small\ttfamily},
  numbers=none,
  numberstyle=\tiny\color{gray},
  keywordstyle=\color{blue},
  commentstyle=\color{dkgreen},
  stringstyle=\color{mauve},
  breaklines=true,
  breakatwhitespace=true,
  tabsize=3
}

\setlength{\parindent}{0pt}
\setlength{\parskip}{1ex}
\setlength{\textwidth}{17cm}
\setlength{\textheight}{24cm}
\setlength{\oddsidemargin}{-.7cm}
\setlength{\evensidemargin}{-.7cm}
\setlength{\topmargin}{-.5in}


\predate{
\begin{center}
}
\postdate{
\\
\vspace{1.5cm}
\includegraphics[scale=0.7]{imag.png}
\end{center}}


\title {{ {\huge Compte rendu du projet}} \\
``{\em Simulation Orientée-Objet de systèmes multiagents}'' }

\author {Equipe 32 \\
\\
DOUMA Nejmeddine\\
GOUTTEFARDE Léo\\
KACHER Ilyes}
\date{Lundi 16 Novembre 2015}
% \date{Lundi 16 Novembre 2015\endgraf\bigskip
% Equipe NN}

\lhead{Projet POO}
\rhead{Compte rendu}

\begin{document}
\pagestyle{fancy}
\maketitle

\begin{center}
\section* {Introduction }
\end{center}


L'objectif de ce TP était d'implémenter en \texttt{Java} les simulations
graphiques de quelques systèmes multiagents, l'un des points les plus importants étant d'aborder les aspects fondamentaux de la programmation orientée objet.

Nous avons implémenté trois simulateurs d'automates cellulaires : le jeu de la vie de Conway, le jeu de l'immigration, le modèle de ségrégation de Schelling, ainsi qu'un simulateur de Boids (système d'essaims auto-organisés).


$\newline$
\section {Choix conceptuels}
\subsection{Organisation en package}
Afin de faciliter la navigation au sein des différentes classes du projet, nous les avons réparties en différents packages :
\begin{itemize}
\item \texttt{element} : classes des différents agents (\texttt{Boid}, \texttt{Lighter}, \texttt{Predator} et \texttt{Prey})
\item \texttt{event} : classes relatives aux événements
\item \texttt{group} : classes qui gèrent les groupes d'agents
\item \texttt{simulator} : classes simulator qui implémentent \texttt{Simulable}
\item \texttt{test} : contient tous les tests
\item \texttt{utility} : contient des classes diverses utilisés par les autres classes du projet
\end{itemize}
\subsection{Un premier simulateur : Balls}
Pour réaliser ce simulateur nous avons suivi le sujet qui a été assez guidé. Cette étape nous a permis de nous familiarisé avec la librairie \texttt{gui.jar} fournie et en particulier l'interface \texttt{Simulable}.
\subsection{Automates cellulaires}
Nous avons cherché à factoriser notre code tout en profitant des avantages de la programmation orientée objet en implémentant deux classes abstraites \texttt{Automaton} et \texttt{AutomatonSimulator}. Pour chaque système d'automate cellulaire simulé il y a une classe concrète qui hérite la classe \texttt{Automaton} et implémente les règles du système, et une classe qui le simule en héritant de \texttt{AutomatonSimulator}. Nous avons remarqué plusieurs similarités entre l'implémentation du jeux de l'immigration et du modèle du Schelling, donc, dans le but de plus factoriser notre code, nous avons définie une autre classe abstraite \texttt{ExtendedAutomaton} qui étend \texttt{Automaton} et qui est héritée par les implémentations de ces deux systèmes.

\begin{center}
\begin{tabular}{|l|c|c|c|}
  \hline
    Système & Implémentation & Simulateur \\
  \hline
   Jeux de la vie de Conway & \texttt{Life} & \texttt{LifeSimulator}\\
  \hline
  Jeux de l'immigration & \texttt{Immigration} & \texttt{ImmigrationSimulator}\\
  \hline
  Modèle de Schelling & \texttt{Schelling} & \texttt{SchellingSimulator}\\
  \hline
 
\end{tabular}
\end{center}

Nous avons aussi généralisé l'utilisation de \texttt{EventManager} pour le calcul des générations suivantes pour tous les systèmes d'automate cellulaire simulés en définissant \texttt{AutomatonEvent} qui hérite la classe abstraite \texttt{Event}.


\subsection{Mouvement d’essaims auto-organisés : les Boids}
\subsubsection {Le système de boids}
Nous avons presque utilisé la même architecture que pour le simulateur des balles. La classe abstraite \texttt{Boid} implémente les règles communes d'un agent boid, la classe \texttt{Boids} implémente un environnement de plusieurs boids tandis que le simulateur du système se trouve dans la classe \texttt{BoidsSimulator}. 

\subsubsection {Les types de boids}

Afin de compléter le comportement d'un boid nous avons ajouté les classes \texttt{Prey}, \texttt{Predator} et \texttt{Lighter}. Ces classes héritent de \texttt{Boid} et implèmentent les règles nécessaires afin d'obtenir un boid proie ou prédateur. Le boid \texttt{Lighter} est seulement un boid classic ayant une apparence différente, il s'agit d'un boid scintillant en forme de cercle.

\subsubsection {Le système Proies/Prédateurs}

Pour répondre à la simulation du système de proies et de prédateurs nous avons ajouté les règles suivantes :
\begin{itemize}
\item Un prédateur chasse la première proie dans son champ de vision au maximum de sa vitesse.
\item Un prédateur mange une proie si elle se trouve à une petite distance du prédateur.
\item Un proie s'enfuit au maximum de sa vitesse dès qu'un prédateur entre dans son champ de vision.
\end{itemize}

\section {Tests}

\subsection {Exécution}
Tous les classes test sont dans le package \texttt{test} et sont repatis selont le tableau suivant:

\begin{center}
\begin{tabular}{|l|c|c|}
  \hline
    Classe & Système simulé & Commande pour l'exécution \\
  \hline
  \texttt{TestGUI} & Test de l'interface graphique & \texttt{make exeGUI}\\
  \hline
   \texttt{TestBalls} & Test demandé à la première question & \\
  \hline
  \texttt{TestBallsSimulator} & Balls & \texttt{make exeBSim}\\
  \hline
  \texttt{TestLifeSimulator} & Jeux de la vie de Conway & \texttt{make exeLifeSim}\\
  \hline
  \texttt{TestImmigrationSimulator} & Jeu de l'immigration & \texttt{make exeImmSim}\\
  \hline
  \texttt{TestSchellingSimulator} & Modèle de Shelling & \texttt{make exeSSim}\\
  \hline
  \texttt{TestEventManager} & Test de l'event manager & \texttt{make exeEvents}\\
  \hline
  \texttt{TestBoidsSimulator} & Boids simples & \texttt{make exeBoids}\\
  \hline
  \texttt{TestPreyPredatorSimulator} & Boids proies/prédateurs  & \texttt{make exePred}\\
  \hline
  
 
\end{tabular}
\end{center}

\subsection {Résultats}
\subsubsection {Jeux de la vie de Conway}
Pour tester ce système, nous avons implémenter certains \textit{pattern} calssiques comme \textit{toad}, \textit{pulsar} ou \textit{blinker} que nous avons récupéré sur cette adresse: 
\newline
\url{https://en.wikipedia.org/wiki/Conway's_Game_of_Life#Examples_of_patterns}
\newline
Nous avons réussi à obtenir les résultats attendus avec le bon nombre d'itérations.
\subsubsection {Jeux de l'immigration}
Nous avons pu constater l'effet des règles du jeux sur les cellules, en particulier, comment un état se propage sur la grille.

\subsubsection {Modèle de Schelling}
Le tableau suivant contient le nombre des itérations nécessaires avant d'obtenir une ségrégation en fonction du seuil \textit{K} et du nombre des couleurs:
\begin{center}
\begin{tabular}{|x{5cm}|x{2cm}|x{2cm}|x{2cm}|x{2cm}|x{2cm}|}\hline
\diag{.1em}{5cm}{\texttt{numberOfColor}}{\textit{K}} & 2 & 4 & 5 & 7 & 8\\ \hline
2 & 55 & 16 & 8 & 0 & 0\\ \hline
7 & +10000 $\infty$ & 409 & 50 & 11 & 0\\ \hline
14 & +10000 $\infty$ & 2363 & 237 & 17 & 0\\ \hline
49 & +10000 $\infty$ & +10000 $\sim$ & 2270 & 40 & 0\\ \hline

\end{tabular}
\end{center}
+10000 $\sim$ : Nous avons dépassé 10000 itération en ayant qu'une ségrégation partielle.
\newline
+10000 $\infty$ : Nous avons dépassé 10000 itération sans aucune ségrégation en cours de formation.
\newline
\newline
Nous constatons que le nombre des itérations nécessaires avant d'obtenir une ségrégation est inversement proportionnel à \textit{K} et proportionnel à \texttt{numberOfColor}.

\subsubsection {Boids}
Sur une même grille, nous avons lancer deux groupe de boids différents: des boids triangulaires (les \texttt{Pray}) et des boids qui sont circulaires (les \texttt{Lighter}). Nous avons réussi à visualiser la cohabitation entre ces deux groupes différents sur une même grille.

\subsubsection {Système Proie/Prédateur}
Les \texttt{Predator} cherchent à attrapper les \texttt{Pray} sans se laisser géner par les \texttt{Lighter} qui partagent la même grille avec eux. On constate bien le rôle joué par l'event manager qui est de gérer le calcule des états suivants des agents à des fréquences différentes.

\begin{center}
\section* {Conclusion }
\end{center}

Ce projet nous a permis de nous confronter aux problématiques du modèle orienté objet et comment profiter de ces aspects fondamenteaux pour produire un code à la fois factorisé et réutilisable. Il nous a permis non seulement de mieux comprendre les notions du cours
mais aussi de découvrir le monde des simulateurs multi-agents. Le
projet était intéressant et il a permis d’avoir un rendu visuel, ce qui est fort agréable.
\end{document}




% 4 pages max
% Explique / justifie choix conception / bonne utilisation des classes et méthodes
% Peut également décrire principaux tests effectués et résultats obtenus

