\documentclass [a4paper,11pt,titlepage] {article}
\usepackage{graphicx}
\usepackage{pdfpages}
\usepackage{fancybox}
\usepackage[francais]{babel}
\usepackage[utf8]{inputenc}
% \usepackage[T1]{fontenc}
\usepackage{amsmath,amsfonts,amssymb}
\usepackage{fancyhdr}
\usepackage{stackrel}
\usepackage{xspace}
\usepackage{url}
\usepackage{titling}

\setlength{\parindent}{0pt}
\setlength{\parskip}{1ex}
\setlength{\textwidth}{17cm}
\setlength{\textheight}{24cm}
\setlength{\oddsidemargin}{-.7cm}
\setlength{\evensidemargin}{-.7cm}
\setlength{\topmargin}{-.5in}


\predate{
\begin{center}
}
\postdate{
\\
\vspace{1.5cm}
\includegraphics[scale=0.7]{imag.png}
\end{center}}


\title {{ {\huge Compte rendu du projet}} \\
``{\em Simulation Orientée-Objet de systèmes multiagents}'' }

\author {Equipe 32 \\
\\
DOUMA Nejmeddine\\
GOUTTEFARDE Léo\\
KACHER Ilyes}
\date{Lundi 16 Novembre 2015}
% \date{Lundi 16 Novembre 2015\endgraf\bigskip
% Equipe NN}

\lhead{Projet POO}
\rhead{Compte rendu}

\begin{document}
\pagestyle{fancy}
\maketitle

\begin{center}
\section* {Introduction }
\end{center}

blabla intro


$\newline$
\section {Choix conceptuels}
\subsection {Jeux de Conway}

blabla


\subsection {Modèle de Schelling}

blabla


\section {Tests}

blabla tests


\end{document}




% 4 pages max
% Explique / justifie choix conception / bonne utilisation des classes et méthodes
% Peut également décrire principaux tests effectués et résultats obtenus

